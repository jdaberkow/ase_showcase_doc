\chapter{Autonomes Ladeverhalten} \label{kap:AutonomesLadeverhalten}

\section{Lokalisation der Ladestation} %TODO: Andi

\section{Einparken in der Ladestation} %TODO: Julian E.

\section{Andocken an die Ladestation} %TODO: Timo M.

- Ausgangssituation: AMiRo steht vorwärts in der Ladestation
- 180$^\circ$ Drehung anhand der Odometrie um die Ladepins zur Station zu richten
- Justierung der Position anhand der Abstandssensoren und der schwarzen Streifen
- Abschalten der PWM und Überwachung, ob sich die Position ändert 
	- Mögliche Fehlerquellen für Positionsänderung: 
		- AMiRo wird durch den Einfluss des Magnetfeldes auf die Schraube gedreht, da die Position noch nicht gut genug justiert ist
		- AMiRo wird durch die Federung der Ladepins von der Station abgestoßen, da die Schraube nicht genau vor dem Magneten war und somit die magnetische Kraft nicht groß genug war 

\section{Ansteuern und Überwachen des Ladevorgangs} %TODO: Timo M.
- Aktivieren des DiWheelDrive-Board Ladepfades (+ kurzes Abwarten)
- Kontrolle, ob mehr als 9V anliegen (wenn nicht -> Position erneut justieren)
- Überwachen des Ladevorgangs
	- Auslesen der Ladestatus der Akkus (\% und Zeit bis geladen)
	- Überwachen der Odometrie 
		- falls sich der AMiRo aufgrund von äußeren Einwirkungen bewegen sollte -> Abbruch des Ladevorgangs und Position erneut justieren

