\chapter{Autonomes Ladeverhalten} \label{kap:AutonomesLadeverhalten}

\section[Lokalisation der Ladestation]{Lokalisation der Ladestation\hfill {\normalsize A.G.}} %TODO: Andi G.

\section[Einparken in der Ladestation]{Einparken in der Ladestation\hfill {\normalsize J.E.}} %TODO: Julian E.

\section[Andocken an die Ladestation]{Andocken an die Ladestation\hfill {\normalsize T.M.}} %TODO: Timo M.
- Ausgangssituation: AMiRo steht vorwärts in der Ladestation
- 180$^\circ$ Drehung anhand der Odometrie um die Ladepins zur Station zu richten
- Justierung der Position anhand der Abstandssensoren und der schwarzen Streifen
- Abschalten der PWM und Überwachung, ob sich die Position ändert 
	- Mögliche Fehlerquellen für Positionsänderung: 
		- AMiRo wird durch den Einfluss des Magnetfeldes auf die Schraube gedreht, da die Position noch nicht gut genug justiert ist
		- AMiRo wird durch die Federung der Ladepins von der Station abgestoßen, da die Schraube nicht genau vor dem Magneten war und somit die magnetische Kraft nicht groß genug war 

\section[Ansteuern und Überwachen des Ladevorgangs]{Ansteuern und Überwachen des Ladevorgangs\hfill {\normalsize T.M.}} %TODO: Timo M.
- Aktivieren des DiWheelDrive-Board Ladepfades (+ kurzes Abwarten)
- Kontrolle, ob mehr als 9V anliegen (wenn nicht -> Position erneut justieren)
- Überwachen des Ladevorgangs
	- Auslesen der Ladestatus der Akkus (\% und Zeit bis geladen)
	- Überwachen der Odometrie 
		- falls sich der AMiRo aufgrund von äußeren Einwirkungen bewegen sollte -> Abbruch des Ladevorgangs und Position erneut justieren

