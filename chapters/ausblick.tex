\chapter{Ausblick} \label{kap:ausblick}

Ausblick Ladeverhalten...

Die Lokalisation der nächsten Kante über die Landmarken sollte in weiteren Tests untersucht werden. Hierbei muss abgeschätzt werden, ob die Positionsabschätzung in diesem Rahmen ausreicht oder zu ungenau wird. Eine starke Abweichung kann die Folge eines qualitativ schlechten Kamerabildes sein. Daraus entsteht eine minimal ungenau Distanz, die wiederum in der Winkelberechnung und der konkreten Positionsberechnung einen sehr hohen Einfluss hat. Des weiteren sind für jegliche Operationen mit der OpenCV Blobdetektion die Parameter eventuell anzupassen, da sie momentan für die Lichtumgebung der Telewerkbank angepasst sind. Die OpenCV Blobdetektion selber liefert lediglich für quadratische Formen einen akzeptablen Radius und Mittelpunkt (benötigt für die Berechnung der relativen Torpfostenmaße in Bildkoordinaten). Für nicht quadratische Formen muss daher eingehend ein anderes Verfahren verwendet werden.

Die momentane Nutzung der Kamera hat eine sehr starke Latenz von bis zu drei Sekunden. Dadurch wird die Objekterkennung über das Kamerabild während einer Bewegung erheblich beeinträchtigt. Aus diesem Grund muss die interne Verarbeitung der Kamera beziehungsweise des Kamerabildes optimiert werden (Linux-Treiber). 

Ein weiterer Punkt ist die Erstellung eines Sensormodelles für jeden eingesetzten AMiRO hinsichtlich der Infrarotsensoren, da nicht jeder Sensor bei gleichen Bedingungen die exakt selben Werte zurückliefert. Dadurch, und mit angepassten Parametern, lassen sich weichere Bewegungen während der Kantenverfolgung und dem Anfahren über die Infrarotbodensensoren erzielen.

Ausblick Szenario Koordination...

Ausblick RSB und Spread Kommunikation...

Ausblick Realisierung der Spielelemente...

Das externe Spieltracking muss in Zukunft weiteren Tests hinsichtlich unterschiedlicher Lichtverhältnisse und Kamerapositionen unterzogen werden. Aktuelle Ergebnisse sind lediglich unter den Bedingungen der Telewerkbank entstanden, was dazu führt, dass die Parameter der Markerdetektion sowie der Blobdetektion an anderen Orten möglicherweise einer Feinjustierung unterzogen werden müssen.
Ferner ist die aktuelle grafische Benutzeroberfläche mithilfe der OpenCV Bibliothek realisiert, welche in ihrem Funktionsumfang sehr eingeschränkt ist. Denkbar wäre an dieser Stelle die Umsetzung der Oberfläche mithilfe des Qt Frameworks, welches die Integration zusätzlicher Features, sowie die Einführung klassicher Interaktion mit Maus und Tastatur erheblich vereinfachen würde.